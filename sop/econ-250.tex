\documentclass{grad-statement}

\addbibresource{refs.bib}

\usepackage{ali-todo}
\usepackage{amsmath, amsfonts}

\setlength{\parskip}{\baselineskip}

\author{Pattison}
\title{Statement of Purpose}

% Describe your reasons for applying to the Department of Economics at Stanford , preparation for this field of study, research interests, future career plans, and any other aspects of your background or interests that may aid the admission committee in evaluating your aptitude and motivation for graduate study. The maximum recommended length is 250 words. 

\begin{document}

\noindent
I'm applying Stanford to become an academic economist conducting research in applied microeconomics and political economics.
Broadly, I aim to study economic opportunity, spatial inequality, and political polarization in the United States.

I graduated \emph{summa cum laude} from Carleton College with a double major in mathematics and statistics.
My math thesis presented an exposition of several seminal results in algebraic number theory and my statistics thesis applied deep learning sequence models to basketball.
I found that these sequentially-aware neural networks provided little improvement over simpler models, suggesting a minimal role of momentum.
I also coauthored an algorithms paper, helped build computational infrastructure for an international physics research group, and spent two stints at the University of Minnesota as an applied cryptography RA.
Since graduating, I've worked as a predoctoral fellow at Opportunity Insights.

In ongoing independent work that I hope to continue in graduate school, I propose a new measure of ideological polarization that captures both disagreement on individual questions and the extent to which opinions correlate across issues \parencite{pattison-2025-polarization}.
Soon, I plan to connect this to the large literature on affective polarization.
For example:
Does my measure correlate across place with indices of affective polarization?
If so, which appears first?
Does sensationalist media coverage cause opinions to diverge.
If so, does this divergence persist after the public discourse moves on?

I aim to answer these questions both theoretically and empirically; Stanford's excellence across both domains makes it the ideal place to launch this agenda.
% Stanford's empirical and theoretical excellence make it the ideal place to launch this agenda.

\printbibliography

\end{document}
