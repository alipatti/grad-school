\documentclass[compact,runin]{grad-statement}

\addbibresource{refs.bib}

\usepackage{ali-todo}
\usepackage{amsmath, amsfonts}

\RequirePackage{xspace}
\newcommand\school{Harvard\xspace}
\newcommand\program{Economics\xspace}

\author{Pattison}
\title{Statement of Purpose}

\begin{document}

\noindent
I am applying to \school to become an academic economist in applied microeconomics and political economy.
Broadly, I aim to study	questions about opportunity, spatial inequality, and growing political polarization in the United States.

\section{Background and Prior Research}

I graduated \emph{summa cum laude} from Carleton College with a double major in mathematics and statistics.
While there, I worked with Professor Nicholas Hopper at the University of Minnesota developing a method for users of private messaging apps to report harmful content without weakening the guarantees of end-to-end encryption.
I presented preliminary work at a major conference and am currently preparing a manuscript for these and related results.

At Carleton, I also served as a computational RA for an international physics collaboration, did theoretical computer science research with Professor David Liben-Nowell, and wrote two senior theses.
For my math major, I presented an exposition of Dirichlet's unit theorem---a seminal result in algebraic number theory.
For my statistics major, I modeled basketball games as sequences of discrete events to which I applied deep-learning techniques originally developed for natural language processing.

During my senior year, I developed an interest in economics after taking an upper-level price theory class where I was impressed by the ability of economic models to describe situations I had thought were not amenable to rigorous study.
I decided to pursue this interest and, since graduating in June 2024, have worked as a predoctoral fellow at Opportunity Insights (OI) on a project exploring disparities in life expectancy across the U.S.
I've appreciated how this work---and OI more generally---uses clever economic and computational techniques to bring quantitative clarity to issues that were not previously well understood.
I hope to continue doing this in my own research.

\section{Research Interests}

My experience growing up in an urban public school system and my more recent work at OI have made me particularly interested in issues of opportunity and spatial inequality.
% Why, and to what degree, do some children have better life outcomes than others?\note{better grounding sentence}
I plan to continue studying this in graduate school.

In particular, I propose to investigate how rising Chinese import competition in the 2000s affected the long-term outcomes of children raised in trade-exposed U.S. communities.
It is well known that this ``China Shock'' had long-lasting negative effects on exposed adults, but the intergenerational impact is unknown \parencite{autor-etal-2025-places-people}.
% \footnote{
% 	``And what are the consequences [for] children who are raised in trade-exposed places?
% 	More work is needed to understand the long-term and intergenerational consequences of trade adjustment.'' -- \textcite[p. 75]{autor-etal-2025-places-people}}
Are kids in textile manufacturing towns worse off than they were 30 years ago?%
\footnote{
	This project will use IRS and Census microdata, which I have worked with extensively in my time at OI.
}

Another line of research has found that social mobility has declined most in places where parental employment rates have also fallen \parencite{chetty-etal-2024-race-trends}, suggesting a link between labor market conditions and intergenerational mobility.
But the direction of causality remains unclear.
Do declining labor markets worsen children's outcomes, or are both symptoms of deeper social decay?

To answer these questions, my ideal experiment first would randomly expose some places to foreign import competition (causing factory closures or mass layoffs) and then randomly assign children to grow up in exposed or unexposed areas.
The difference in their adult outcomes would reveal the causal effect of import competition on children's prospects.
%
To approximate random import exposure, I will use the classic ``China Shock'' shift-share instrument \parencite[for example]{autor-etal-2025-places-people}.
Intuitively, places where employment was concentrated in sectors that saw large Chinese supply increases (textiles) were more affected than those where employment was concentrated in unexposed industries (food processing).

But this approach alone is insufficient: If high-ability families are more likely to leave an area when work disappears, then simply comparing outcomes before and after will overstate the China Shock's effect.
%
To address this residential selection, I take inspiration from \textcite{chetty-etal-2024-race-trends} and leverage variation in the timing and destination of moves to identify yearly causal effects of each commuting zone on children's outcomes. 
I can then use these on the left-hand-side of an event study or first-differences regression to discern the China Shock's true impact.%
\footnote{
	For power, I could also incorporate an event study into \autoref{eq:movers} to estimate everything in a single step.
}

To illustrate my identification strategy, consider two children born to similar families in Cleveland in 1985, one of whom moves to Minneapolis in 2000 and the other in 2001.
Then, the difference in their mean outcomes reflects both the causal effect of spending 2000 in Minneapolis rather than Cleveland \emph{and} potential differences between families that move in different years.
To account for selection into moving, I include yearly ``push'' and ``pull'' factors that allow for time-varying selection into both the choice to move and the choice of destination.
I package all such comparisons into the following regression estimated on all one-time movers:
\begin{equation}
	\label{eq:movers}
	y_i =
	\alpha_{o(i),d(i)}
	% + \underbrace{\pi_{om} + \rho_{dm}}_{\substack{\text{push and pull} \\ \text{factors}}}
	+ \pi_{o(i),m(i)} + \rho_{d(i),m(i)}
	% + \sum_{t = c}^{c + 22} \mu_{z(i, t), t}
	+ \sum_{t = c(i)}^{m(i) - 1} \mu_{o(i), t}
	+ \sum_{t = m(i)}^{c(i) + 22} \mu_{d(i), t}
	+ X_i \gamma
	+ \varepsilon_i.
\end{equation}
Here, $y_i$ is the income rank of individual $i$ at age 28 who moves from $o(i)$ to $d(i)$ in year $m(i)$ and $X_i$ includes characteristics such as birth cohort $c(i)$ and parental income, occupation, and labor-force status.%
\footnote{
	For identifiability, I also impose a yearly sum-to-zero constraint $\mathbb E(\mu_{zt} \ | \ t) = 0 \; \forall \; t$.}
%
The fixed effects allow for dynamic selection into both the choice to move and the choice of destination, and for selection into particular origin-destination pairs.
Identifying $\{\mu_{zt}\}$---the causal effects of spending year $t$ in place $z$---requires assuming that this selection into specific origin-destination pairs is static.\footnote{
	This differs from the classic age-at-move orthogonality assumption: To identify yearly place effects, my design requires variation in time of move within a cohort \emph{or} in cohort within a fixed time-of-move. 
	For the first source of variation, age at move coincides with year of move; for the second, with cohort.
	There's likely selection on both.
}

Next, I will study heterogeneity:
Were children of high-income professionals affected differently than children of unskilled workers?
% https://www.nber.org/papers/w33121
Did children who left fare better than those who stayed?
Were places that retained high-income residents less affected or quicker to recover?
(Recent unpublished work by \textcite{card-rothstein-2025-neighborhoods} indicates that composition explains a large part of a place's causal effect, suggesting a peer-effects mechanism at play.)
% https://www.brookings.edu/wp-content/uploads/2021/09/15985-BPEA-BPEA-FA21_WEB_Autor-et-al_Glaeser-Comment.pdf

Finally, this project will serve as a jumping-off point for a broader research agenda about spatial inequality, the places ``left behind'' by globalization, and growing tensions in American politics.
For example, in ongoing independent work \parencite{pattison-2025-polarization}, I develop a measure of public ideological polarization and show that it has been steadily increasing since the 1990s.
Surprisingly, most of this increase has been driven by growing polarization \emph{within} regions of the country and \emph{within} urban and rural areas rather than growing disagreement between places.
If spatial divides don't explain diverging views, what other factors are at play?
I hope to continue answering these sorts of questions throughout graduate school and beyond.

\printbibliography

\end{document}
