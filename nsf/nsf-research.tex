\documentclass{grfp}

\addbibresource{refs.bib}

\begin{document}

\noindent
Childhood environment matters.
A large literature shows that (1) adults' incomes depend causally on where they lived as children and (2) these causal effects vary greatly from place to place.
Recently, \textcite{chetty-etal-2024-race-trends} show that these causal ``place effects'' themselves change across cohorts and that the changes closely track changes in local parental employment rates. This suggests a link between local labor market conditions and intergenerational mobility, but the direction of causality remains unclear.
Do declining labor markets worsen children's outcomes, or are both symptoms of deeper social decay?
%
This project will improve understanding of how negative labor demand shocks affect children's outcomes by providing the first estimates of how rising import competition from China impacted children raised in affected areas.
% This large and long-term negative effects for adults, but very little is known about its impact on children.
Recent work by \textcite{autor-etal-2025-places-people} explicitly notes this as an open question.

My ideal experiment would first randomly expose some places to foreign import competition (causing factory closures or mass layoffs) and then randomly assign children to grow up in exposed or unexposed areas.
The difference in their adult outcomes would reveal the causal effect of import competition on children's prospects.
%
To approximate the first step, I use the instrument proposed by \textcite{autor-etal-2025-places-people} to isolate exogenous variation in area-level exposure to import competition in the decade following China's 2001 admission to the World Trade Organization.
Intuitively, places where employment was concentrated in exposed sectors such as textiles were more affected than those where employment was concentrated in unexposed industries like meatpacking.
%
To simulate the second randomization step and address the possibility that higher-potential families may depart an area more readily when jobs disappear, I will leverage quasi-experimental variation in the timing and destinations of moves to separate compositional changes from shifts in places' causal effects.
My analysis will answer
\begin{enumerate}
	\itshape
	\item How does globalization affect the long-term outcomes of children in affected communities?
	\item To what extent do these effects reflect sorting versus genuine changes in opportunity?
	\item What places and people suffer most and why?
\end{enumerate}

\section{Data}

The main data source for this project will be individual-level IRS and U.S. Census microdata, which I have used extensively in my work at Opportunity Insights.
These data
% are described extensively in \textcite{chetty-etal-2024-race-trends} and
allow me to (1) observe yearly residence and income for each tax-filer and their family and (2) use dependent claiming to link parents to children.
My outcome of interest will be children's household income at age 28, and I will focus on the 1978--1997 cohorts.
%
In addition to this microdata, I will use public data on Chinese imports and industry composition at the commuting zone (CZ) level.

\section{Intellectual Merit}

The backbone of my empirical strategy is the standard ``China Shock'' regression \cite{autor-etal-2025-places-people} that connects changes in some outcome $\Delta y_{zh}$ to changes in local labor market exposure to imports $\Delta E_{zh}$, with changes measured between 2000 and some horizon $h$:
\begin{equation}
	\label{eq:first-diffs}
	\Delta y_{zh}
	= \alpha_h
	+ \beta_h \Delta E_{zh}
	+ X_z \gamma
	+ \epsilon_{zh}.
\end{equation}
Here, $z$ indexes CZ, $X_z$ is a vector of covariates, and $\beta_h$ is the parameter of interest.
As is standard, I will construct my local exposure measure as a weighted sum of Chinese import penetration over industries with weights given by local industry employment shares.
This is endogenous (it's determined in part by American demand), so we isolate an exogenous component as in \textcite{autor-etal-2025-places-people}
by lagging employment shares to 1990 and replacing the import penetration in the U.S. with the average import penetration of China in several other highly developed countries.
% via the shift-share instrument
% \begin{equation}
% 	\Delta I_z = \sum_i s_{zi}^\text{(1990)} \cdot \Delta \text{IP}^\text{(other)}_i
% \end{equation}
% where $s_{zi}^\text{(1990)}$ is the local employment share in CZ $z$ from industry $i$ in 1990 and $\Delta \text{IP}^\text{(other)}_i$ is the average import penetration of China in industry $i$ to several other highly developed countries.
Intuitively, this instrument captures supply-driven variation distinct from American demand-side factors, and the literature provides many tests of its validity that I can apply in my work.
% To further control for endogeneity, $X_z$ will include pre-period demographics and manufacturing employment shares so identification comes from variation in the \emph{type} of manufacturing, not merely its presence.

By itself, using this instrument with 2SLS is sufficient to answer our first research question by setting the left-hand side of \autoref{eq:first-diffs} to the change in mean adult income among children living in $z$ in 2000 and those there in year $h$.
However, this estimate is biased by selective migration: Families in $z$ before the shock were likely different---observably and unobservably---from those living in $z$ after.
% For example, they are much more likely to be unemployed or have worse wages.
To capture observable differences, we can repeat this same 2SLS regression after projecting off variation in children's adult income due to observable characteristics like parental education and pre-period parental wages.

To purge selection on unobservables and isolate changes in the causal effect of each CZ, we adapt the framework of \textcite{chetty-etal-2024-race-trends} to allow for place effects that vary over time.
To illustrate our identification, consider two children born to similar families in Detroit in 1985, one of whom moves to Minneapolis in 2000 and the other in 2001.
Then, the difference in their mean outcomes is due to both the causal effect of spending 2000 in Minneapolis rather than Detroit \emph{and} the fact that families moving in different years may be differentially selected.
We model this selection as stemming from time-varying push and pull factors and package all such comparisons into the following regression estimated on all one-time movers:
\begin{equation}
	\label{eq:movers}
	y_i =
	\alpha_{od}
	% + \underbrace{\pi_{om} + \rho_{dm}}_{\substack{\text{push and pull} \\ \text{factors}}}
	+ \pi_{om} + \rho_{dm}
	+ \sum_{t = c}^{c + 22} \mu_{z(i, t), t}
	+ X_i \gamma
	+ \epsilon_i
\end{equation}
where $y_i$ is the income rank of individual $i$ at age 28 who moves from $o$ to $d$ in year $m$, and $X_i$ is a vector of covariates (e.g. cohort, parental age, income, and occupation).
Of interest are the parameters $\set{\mu_{zt}}$: the causal effects of spending year $t$ in place $z$.
%
For OLS to recover $\set{\mu_{zt}}$, we require (1) the well-validated assumption that age at move is orthogonal to unobserved family inputs and (2) that there is no dynamic selection into particular origin-destination pairs. In other words, we require that variation not absorbed by the push and pull factors $\pi_{om}$ and $\rho_{dm}$ is idiosyncratic.
% We can test this assumption by XXXX.
For identifiability, we also impose a yearly sum-to-zero constraint $\mathbb E(\mu_{zt} \given t) = 0 \; \forall \; t$.
%
Setting $\Delta y_{zh} = \mu_{zh} - \mu_{z,2000}$ (the change in the causal effect of each CZ) and estimating \autoref{eq:first-diffs} answers our second research question.
% For clarity, I have presented this as two separate regressions, but it could be combined into a single estimation to improve precision.

% To see the role of peer effects, I will regress the residualized changes in CZ causal effects on changes in composition to see if stayers benefit from the retention of high-ability peers.
% These exercises will provide valuable insight into the most effective policies for helping areas recover.

Finally, I will examine heterogeneity in these effects at the individual and place level.
For example, do children of managers and other elites fare better in the aftermath than those whose parents are unskilled workers?
% https://www.nber.org/papers/w33121
Do leavers fare better than stayers?
I will also assess the degree to which changes in place effects were driven by compositional changes, for example, if CZs that retained high-income or highly educated residents were less affected or recovered more quickly.
Together, these analyses will help clarify the exact policies each community requires.

\section{Broader Impacts}

This project addresses a fundamental policy question:
When local economies decline, do areas suffer because opportunity itself erodes, or because families with greater potential leave?
If cross-sectional outcomes worsen only because of selective out-migration, then policies providing incentives to live in the area may restore opportunity without directly altering local labor markets.
But if causal place effects deteriorate as labor markets contract, then restoring opportunity may require more substantial economic intervention like job creation.
The results will also inform current debates on trade policy by quantifying how the gains and losses from globalization are transmitted across generations.

Beyond its policy relevance, this project will generate a new dataset of yearly causal place effects at the CZ level, which I will make publicly available to facilitate future work on intergenerational mobility, the spatial distribution of opportunity, and the long-term social costs of trade adjustment.

\printbibliography

\end{document}
