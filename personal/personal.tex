\documentclass{grad-statement}

\usepackage{ali-todo}

\title{Personal Statement}
\author{Pattison}

\usepackage{xstring}
\newcommand{\version}{short}

\begin{document}

\IfStrEq{\version}{short}{\noindent}{\section{Personal Background}}%
%
As a teen, I spent five years in an accelerated math program at the University of Minnesota that covered the core sequence of a typical undergraduate math degree.
The program's rigor stood in sharp contrast to the classroom experience at my public high school where two-thirds of the student body failed to meet state reading standards and three-quarters received federal free or reduced price lunch.
I fell in love with math and entered college thinking that nothing else approached the world with sufficient precision.

But after taking several upper-level math classes, I slowly became disillusioned with the prospect of becoming a mathematician and spending my life proving theorems understandable only by a handful of people in my subfield.
Around the same time, my former high school track teammate was shot to death in a mall, and another of my close high school friends became a teen parent.
%
I began to question the social forces that had set us on such drastically different trajectories, but the methods taught in introductory social science classes felt woefully underpowered to grapple with such complicated issues.
I was faced with a conundrum:
Fields I found rigorous studied problems I found unsatisfyingly abstract.
Fields studying problems that moved me lacked sufficient rigor.

Gradually, I realized that the choice between rigor and relevance was a false dichotomy.
During my senior year, I took a price theory class that opened my eyes to economics as a lens through which to study the inequality so apparent to me during my years in public school.
My work at Opportunity Insights is a continuation of this thread.
There, I've had the privilege of tackling issues that I find personally meaningful---education, social mobility, health inequality---with the exactness that first drew me to mathematics.

My individual work and the surrounding research environment have been intellectually challenging (how do you estimate standard errors for millions of parameters when no closed-form solution exists?) and personally fulfilling (how do we develop policy that enables children to rise out of poverty?).
%
\IfStrEq{\version}{short}{%
	I hope to continue doing research that checks both of these boxes throughout my Ph.D. and beyond.
}{%
	I've also gained a deeper appreciation for the ingredients that foster this kind of research.
	There are two that I find particularly important: cohesive teams and good tools.
	% I'm invested in making sure the people around me have both.

	\section{Teams}



	As a kid, I was a part of several communities where I felt completely at home.
	At the time, those environments felt unforced and organic, but I’ve since realized that the sense of community I felt was the result of deliberate leadership that I’ve tried to replicate as I’ve gotten older.
	%
	For example, on a backcountry canoeing trip I led while working at a camp over one summer in college, I made special efforts to ensure that a particularly anxious and homesick camper felt like he belonged.
	I sat and talked with him for hours each day, and I later heard that he returned the following summer explicitly because of me.
	I apply this philosophy to every group I'm in.

	At Carleton, I was elected captain of the varsity track team despite being far from the fastest runner.
	%
	Humor was one tool I used.
	I organized traditions like ``room tours'' (a circuit of team members' dorms during which everyone was required to wear a ridiculous hat) and ``poemajj'' (a pre-meet open mic where we parodied songs to prepare for the annual face-off against our crosstown rivals).
	Even something as simple as a well-timed joke during a cold, rainy long run can completely pivot a miserable situation into one that builds group identity.
	This strategy produced tangible results.
	My senior year, I led the mid-distance team to a conference 4x800m relay championship for the first time in nearly a decade.

	In my role as a TA, I encouraged those who had previously been dissuaded from math and computer science by their reputation as overly technical and disconnected from reality.
	I was told by several students that they came only to my office hours because they found my teaching style more effective than other course staff.
	%
	In my current job, I have continued this interest by helping spearhead the development of training materials and leading several Python and Git crash courses to help our new hires learn the coding skills necessary for empirical research.

	Throughout my life, I've valued leaders and teachers whose infectious enthusiasm pulled the best out of me.
	I hope to provide this same mentorship to the next generation of students during graduate school and throughout my career as a Professor of Economics.

	\section{Tools}

	Another way to improve a team is to give it better tools.

	In my work at OI, I'm often frustrated by how programming languages that provide powerful statistical modeling---like Stata and R---are painfully slow on big datasets and awkward to use when managing large complex codebases.
	Meanwhile, tools that work well at scale---like Python and its excellent Polars data frame library---lack good statistical modeling software.
	Switching between languages creates an annoying friction point.

	Over the past year, I've worked to fill this gap by writing a statistical library for Polars on par with the capabilities of R and Stata.
	I've implemented a variety of features including generalized linear models that allow for sample weights, regularization, several spline bases, and common sandwich covariance estimators.
	I'm currently working to add random effects, REML hyper-parameter selection, and the ability to absorb high-dimensional fixed effects.
	I plan to continue developing this library throughout graduate school and hope that it soon removes many of the headaches that come with writing and maintaining the codebase of a complicated research project.

	\bigskip

	Throughout my Ph.D. and beyond, I aim to answer questions about inequality and social mobility with the same precision that first drew me to mathematics and computer science.
	I have strong experience building environments---social and technical---in which this sort of work can be done.
}

\end{document}
