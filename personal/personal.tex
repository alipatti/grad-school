\documentclass[berkeley]{grad-statement}

\usepackage{ali-todo}

\title{Personal Statement}
\author{Pattison}

\newcommand{\version}{long}

\begin{document}

\noindent
As a teen, I spent five years in an accelerated math program at the University of Minnesota that covered the core of a typical undergraduate math degree.
Its rigor stood in sharp contrast to the classroom experience at my public high school where two-thirds of students failed to meet state reading standards and three-quarters received federal free or reduced-price lunch.
I fell in love with math and entered college thinking I'd be a mathematician.

But after taking several upper-level math classes, I became disillusioned with the idea of spending my life proving theorems understandable only by a handful of specialists.
Around the same time, a former high school track teammate was shot to death, and another close high school friend became a teen mother.
I began to question the social forces that had set us on such different trajectories, but the methods taught in introductory social science classes felt woefully underpowered to grapple with such complicated issues.
I faced a conundrum:
Fields I found rigorous studied problems I found unsatisfyingly abstract.
Fields studying problems that moved me lacked sufficient rigor.

Gradually, I realized that the choice between rigor and relevance was a false dichotomy, and in my senior year, a price theory class opened my eyes to economics as a lens through which to study the inequality so apparent in my public schooling.
At Opportunity Insights, I've had the privilege of tackling issues I find personally meaningful---education, social mobility, health inequality---with the exactness that first drew me to mathematics.

My individual work and the research environment have been intellectually challenging (how do we efficiently bootstrap millions of standard errors?) and personally fulfilling (how do we develop policies that enable children to rise from poverty?).
I hope to continue doing research that checks both these boxes.

\IfStrEq{\version}{short}{}{

	\medskip

	My time at OI has also given me a deeper appreciation for the ingredients that foster this kind of research.
	There are two that I find particularly important: cohesive teams and good tools.
	% I'm invested in making sure the people around me have both.

	\section{Teams}

	As a kid, I was a part of several communities where I felt completely at home.
	At the time, those environments felt unforced and organic, but I’ve since realized that the camaraderie I felt was the result of deliberate leadership that I’ve tried to replicate as I’ve gotten older.

	For example, on a backcountry canoeing trip I led while working at a camp over one summer in college, I made special efforts to ensure that a particularly anxious and homesick camper felt like he belonged.
	I sat and talked with him for hours each day, and I later heard that he returned the following summer explicitly because of me.
	At Carleton, I used my role as a TA to encourage those who had been discouraged by the reputation of math and computer science as technical and disconnected from reality.
	I was told by several students that they came only to my office hours because they found my teaching style more effective than other course staff.

	In my junior year, I was elected captain of the varsity track team;
	I wasn't the fastest, but I was one of the funniest.
	Even something as simple as a well-timed joke during a cold, rainy long run can completely pivot a miserable situation into one that builds group identity.
	In my current job, I've continued this interest in team-building and mentorship by helping spearhead the development of training materials and leading several Python and Git crash courses to help our new hires learn the coding skills necessary for empirical research.
	I make my computer science expertise available to my coworkers as a resource in their work.

	Throughout my life, I've valued leaders and teachers whose infectious enthusiasm lit a fire under me and made me achieve things that I thought I wasn't capable of.
	I hope to provide this same mentorship to the next generation of students during graduate school and throughout my career as a professor.

	\section{Tools}

	Another way to improve a team is to give them better tools.
	I have a strong commitment to producing these types of public goods.

	In my work at OI, I'm often frustrated by how programming languages that provide powerful statistical modeling---like Stata and R---are painfully slow on big datasets and awkward to use when managing large complex codebases.
	Meanwhile, tools that work well at scale---like Python and its excellent Polars data frame library---lack good statistical modeling software.
	Switching between languages creates an annoying friction point and increases the chances of bugs.

	Over the past year, I've worked to fill this gap by writing a statistical library for Polars on par with the capabilities of R and Stata.
	So far, I've implemented features including OLS and generalized linear models that allow for sample weights, regularization, several spline bases, and common sandwich covariance estimators.
	I'm currently working to add random effects, REML hyper-parameter selection, and the ability to absorb high-dimensional fixed effects.
	I plan to continue developing this library throughout graduate school and hope that it soon removes many of the headaches that come with a complicated empirical research project.

	\bigskip

	Throughout my Ph.D. and beyond, I aim to answer economic questions with the same precision that first drew me to mathematics and computer science.
	I have strong experience building environments---social and technical---in which this sort of work can be done.
}

\end{document}
